\documentclass[]{article}
\usepackage{lmodern}
\usepackage{amssymb,amsmath}
\usepackage{ifxetex,ifluatex}
\usepackage{fixltx2e} % provides \textsubscript
\ifnum 0\ifxetex 1\fi\ifluatex 1\fi=0 % if pdftex
  \usepackage[T1]{fontenc}
  \usepackage[utf8]{inputenc}
\else % if luatex or xelatex
  \ifxetex
    \usepackage{mathspec}
  \else
    \usepackage{fontspec}
  \fi
  \defaultfontfeatures{Ligatures=TeX,Scale=MatchLowercase}
\fi
% use upquote if available, for straight quotes in verbatim environments
\IfFileExists{upquote.sty}{\usepackage{upquote}}{}
% use microtype if available
\IfFileExists{microtype.sty}{%
\usepackage{microtype}
\UseMicrotypeSet[protrusion]{basicmath} % disable protrusion for tt fonts
}{}
\usepackage[margin=1.5cm]{geometry}
\usepackage{hyperref}
\hypersetup{unicode=true,
            pdftitle={Multinomial Logistic Regression},
            pdfborder={0 0 0},
            breaklinks=true}
\urlstyle{same}  % don't use monospace font for urls
\usepackage{color}
\usepackage{fancyvrb}
\newcommand{\VerbBar}{|}
\newcommand{\VERB}{\Verb[commandchars=\\\{\}]}
\DefineVerbatimEnvironment{Highlighting}{Verbatim}{commandchars=\\\{\}}
% Add ',fontsize=\small' for more characters per line
\usepackage{framed}
\definecolor{shadecolor}{RGB}{248,248,248}
\newenvironment{Shaded}{\begin{snugshade}}{\end{snugshade}}
\newcommand{\AlertTok}[1]{\textcolor[rgb]{0.94,0.16,0.16}{#1}}
\newcommand{\AnnotationTok}[1]{\textcolor[rgb]{0.56,0.35,0.01}{\textbf{\textit{#1}}}}
\newcommand{\AttributeTok}[1]{\textcolor[rgb]{0.77,0.63,0.00}{#1}}
\newcommand{\BaseNTok}[1]{\textcolor[rgb]{0.00,0.00,0.81}{#1}}
\newcommand{\BuiltInTok}[1]{#1}
\newcommand{\CharTok}[1]{\textcolor[rgb]{0.31,0.60,0.02}{#1}}
\newcommand{\CommentTok}[1]{\textcolor[rgb]{0.56,0.35,0.01}{\textit{#1}}}
\newcommand{\CommentVarTok}[1]{\textcolor[rgb]{0.56,0.35,0.01}{\textbf{\textit{#1}}}}
\newcommand{\ConstantTok}[1]{\textcolor[rgb]{0.00,0.00,0.00}{#1}}
\newcommand{\ControlFlowTok}[1]{\textcolor[rgb]{0.13,0.29,0.53}{\textbf{#1}}}
\newcommand{\DataTypeTok}[1]{\textcolor[rgb]{0.13,0.29,0.53}{#1}}
\newcommand{\DecValTok}[1]{\textcolor[rgb]{0.00,0.00,0.81}{#1}}
\newcommand{\DocumentationTok}[1]{\textcolor[rgb]{0.56,0.35,0.01}{\textbf{\textit{#1}}}}
\newcommand{\ErrorTok}[1]{\textcolor[rgb]{0.64,0.00,0.00}{\textbf{#1}}}
\newcommand{\ExtensionTok}[1]{#1}
\newcommand{\FloatTok}[1]{\textcolor[rgb]{0.00,0.00,0.81}{#1}}
\newcommand{\FunctionTok}[1]{\textcolor[rgb]{0.00,0.00,0.00}{#1}}
\newcommand{\ImportTok}[1]{#1}
\newcommand{\InformationTok}[1]{\textcolor[rgb]{0.56,0.35,0.01}{\textbf{\textit{#1}}}}
\newcommand{\KeywordTok}[1]{\textcolor[rgb]{0.13,0.29,0.53}{\textbf{#1}}}
\newcommand{\NormalTok}[1]{#1}
\newcommand{\OperatorTok}[1]{\textcolor[rgb]{0.81,0.36,0.00}{\textbf{#1}}}
\newcommand{\OtherTok}[1]{\textcolor[rgb]{0.56,0.35,0.01}{#1}}
\newcommand{\PreprocessorTok}[1]{\textcolor[rgb]{0.56,0.35,0.01}{\textit{#1}}}
\newcommand{\RegionMarkerTok}[1]{#1}
\newcommand{\SpecialCharTok}[1]{\textcolor[rgb]{0.00,0.00,0.00}{#1}}
\newcommand{\SpecialStringTok}[1]{\textcolor[rgb]{0.31,0.60,0.02}{#1}}
\newcommand{\StringTok}[1]{\textcolor[rgb]{0.31,0.60,0.02}{#1}}
\newcommand{\VariableTok}[1]{\textcolor[rgb]{0.00,0.00,0.00}{#1}}
\newcommand{\VerbatimStringTok}[1]{\textcolor[rgb]{0.31,0.60,0.02}{#1}}
\newcommand{\WarningTok}[1]{\textcolor[rgb]{0.56,0.35,0.01}{\textbf{\textit{#1}}}}
\usepackage{graphicx,grffile}
\makeatletter
\def\maxwidth{\ifdim\Gin@nat@width>\linewidth\linewidth\else\Gin@nat@width\fi}
\def\maxheight{\ifdim\Gin@nat@height>\textheight\textheight\else\Gin@nat@height\fi}
\makeatother
% Scale images if necessary, so that they will not overflow the page
% margins by default, and it is still possible to overwrite the defaults
% using explicit options in \includegraphics[width, height, ...]{}
\setkeys{Gin}{width=\maxwidth,height=\maxheight,keepaspectratio}
\IfFileExists{parskip.sty}{%
\usepackage{parskip}
}{% else
\setlength{\parindent}{0pt}
\setlength{\parskip}{6pt plus 2pt minus 1pt}
}
\setlength{\emergencystretch}{3em}  % prevent overfull lines
\providecommand{\tightlist}{%
  \setlength{\itemsep}{0pt}\setlength{\parskip}{0pt}}
\setcounter{secnumdepth}{0}
% Redefines (sub)paragraphs to behave more like sections
\ifx\paragraph\undefined\else
\let\oldparagraph\paragraph
\renewcommand{\paragraph}[1]{\oldparagraph{#1}\mbox{}}
\fi
\ifx\subparagraph\undefined\else
\let\oldsubparagraph\subparagraph
\renewcommand{\subparagraph}[1]{\oldsubparagraph{#1}\mbox{}}
\fi

%%% Use protect on footnotes to avoid problems with footnotes in titles
\let\rmarkdownfootnote\footnote%
\def\footnote{\protect\rmarkdownfootnote}

%%% Change title format to be more compact
\usepackage{titling}

% Create subtitle command for use in maketitle
\providecommand{\subtitle}[1]{
  \posttitle{
    \begin{center}\large#1\end{center}
    }
}

\setlength{\droptitle}{-2em}

  \title{Multinomial Logistic Regression}
    \pretitle{\vspace{\droptitle}\centering\huge}
  \posttitle{\par}
    \author{}
    \preauthor{}\postauthor{}
    \date{}
    \predate{}\postdate{}
  
\usepackage{booktabs}
\usepackage{multicol}

\begin{document}
\maketitle

\hypertarget{example-vertebral-column}{%
\subsection{Example: Vertebral Column}\label{example-vertebral-column}}

Our data example uses ``six biomechanical attributes derived from the
shape and orientation of the pelvis and lumbar spine'' for a patient to
classify the patient into one of three groups representing different
conditions that may be affecting their spine: ``DH'' for disk hernia,
``SL'' for Spondylolisthesis, or ``NO'' for normal (neither of the other
two conditions).

The data are available at
\url{https://archive.ics.uci.edu/ml/datasets/Vertebral+Column} and were
discussed in:

Berthonnaud, E., Dimnet, J., Roussouly, P. \& Labelle, H. (2005).
`Analysis of the sagittal balance of the spine and pelvis using shape
and orientation parameters', Journal of Spinal Disorders \& Techniques,
18(1):40â€``47.

\hypertarget{reading-the-data-in-preprocessing-traintest-split}{%
\paragraph{Reading the data in, preprocessing, train/test
split}\label{reading-the-data-in-preprocessing-traintest-split}}

\begin{Shaded}
\begin{Highlighting}[]
\KeywordTok{library}\NormalTok{(readr)}
\KeywordTok{library}\NormalTok{(purrr)}
\KeywordTok{library}\NormalTok{(dplyr)}
\KeywordTok{library}\NormalTok{(ggplot2)}
\KeywordTok{library}\NormalTok{(gridExtra)}
\KeywordTok{library}\NormalTok{(rpart)}
\KeywordTok{library}\NormalTok{(caret)}

\NormalTok{vertebral_column <-}\StringTok{ }\KeywordTok{read_table2}\NormalTok{(}\StringTok{"http://www.evanlray.com/data/UCIML/vertebral_column/column_3C.dat"}\NormalTok{,}
  \DataTypeTok{col_names =} \OtherTok{FALSE}\NormalTok{)}
\KeywordTok{names}\NormalTok{(vertebral_column) <-}\StringTok{ }\KeywordTok{c}\NormalTok{(}\KeywordTok{paste0}\NormalTok{(}\StringTok{"X_"}\NormalTok{, }\DecValTok{1}\OperatorTok{:}\DecValTok{6}\NormalTok{), }\StringTok{"type"}\NormalTok{)}

\NormalTok{vertebral_column <-}\StringTok{ }\NormalTok{vertebral_column }\OperatorTok
\StringTok{  }\KeywordTok{mutate}\NormalTok{(}
    \DataTypeTok{type =} \KeywordTok{factor}\NormalTok{(type)}
\NormalTok{  )}

\KeywordTok{set.seed}\NormalTok{(}\DecValTok{723}\NormalTok{)}

\CommentTok{# Train/test split}
\NormalTok{tt_inds <-}\StringTok{ }\NormalTok{caret}\OperatorTok{::}\KeywordTok{createDataPartition}\NormalTok{(vertebral_column}\OperatorTok{$}\NormalTok{type, }\DataTypeTok{p =} \FloatTok{0.7}\NormalTok{)}
\NormalTok{train_set <-}\StringTok{ }\NormalTok{vertebral_column }\OperatorTok\StringTok{ }\KeywordTok{slice}\NormalTok{(tt_inds[[}\DecValTok{1}\NormalTok{]])}
\NormalTok{test_set <-}\StringTok{ }\NormalTok{vertebral_column }\OperatorTok\StringTok{ }\KeywordTok{slice}\NormalTok{(}\OperatorTok{-}\NormalTok{tt_inds[[}\DecValTok{1}\NormalTok{]])}
\end{Highlighting}
\end{Shaded}

\hypertarget{fit-and-test-set-classification-error-rate-via-multinomial-logistic-regression}{%
\paragraph{Fit and test set classification error rate via multinomial
logistic
regression}\label{fit-and-test-set-classification-error-rate-via-multinomial-logistic-regression}}

Fit model

\begin{Shaded}
\begin{Highlighting}[]
\NormalTok{multilogistic_fit <-}\StringTok{ }\KeywordTok{train}\NormalTok{(}
\NormalTok{  type }\OperatorTok{~}\StringTok{ }\NormalTok{.,}
  \DataTypeTok{data =}\NormalTok{ train_set,}
  \DataTypeTok{trace =} \OtherTok{FALSE}\NormalTok{,}
  \DataTypeTok{method =} \StringTok{"multinom"}\NormalTok{,}
  \DataTypeTok{trControl =} \KeywordTok{trainControl}\NormalTok{(}\DataTypeTok{method =} \StringTok{"none"}\NormalTok{)}
\NormalTok{)}
\end{Highlighting}
\end{Shaded}

Test set performance: looking for low test set error rate, high test set
accuracy

\begin{Shaded}
\begin{Highlighting}[]
\KeywordTok{mean}\NormalTok{(test_set}\OperatorTok{$}\NormalTok{type }\OperatorTok{!=}\StringTok{ }\KeywordTok{predict}\NormalTok{(multilogistic_fit, test_set))}
\end{Highlighting}
\end{Shaded}

\begin{verbatim}
## [1] 0.1397849
\end{verbatim}

\begin{Shaded}
\begin{Highlighting}[]
\KeywordTok{mean}\NormalTok{(test_set}\OperatorTok{$}\NormalTok{type }\OperatorTok{==}\StringTok{ }\KeywordTok{predict}\NormalTok{(multilogistic_fit, test_set))}
\end{Highlighting}
\end{Shaded}

\begin{verbatim}
## [1] 0.8602151
\end{verbatim}


\end{document}
