\documentclass[]{article}
\usepackage{lmodern}
\usepackage{amssymb,amsmath}
\usepackage{ifxetex,ifluatex}
\usepackage{fixltx2e} % provides \textsubscript
\ifnum 0\ifxetex 1\fi\ifluatex 1\fi=0 % if pdftex
  \usepackage[T1]{fontenc}
  \usepackage[utf8]{inputenc}
\else % if luatex or xelatex
  \ifxetex
    \usepackage{mathspec}
  \else
    \usepackage{fontspec}
  \fi
  \defaultfontfeatures{Ligatures=TeX,Scale=MatchLowercase}
\fi
% use upquote if available, for straight quotes in verbatim environments
\IfFileExists{upquote.sty}{\usepackage{upquote}}{}
% use microtype if available
\IfFileExists{microtype.sty}{%
\usepackage{microtype}
\UseMicrotypeSet[protrusion]{basicmath} % disable protrusion for tt fonts
}{}
\usepackage[margin=1.5cm]{geometry}
\usepackage{hyperref}
\hypersetup{unicode=true,
            pdftitle={Ensemble Methods for Classification},
            pdfborder={0 0 0},
            breaklinks=true}
\urlstyle{same}  % don't use monospace font for urls
\usepackage{color}
\usepackage{fancyvrb}
\newcommand{\VerbBar}{|}
\newcommand{\VERB}{\Verb[commandchars=\\\{\}]}
\DefineVerbatimEnvironment{Highlighting}{Verbatim}{commandchars=\\\{\}}
% Add ',fontsize=\small' for more characters per line
\usepackage{framed}
\definecolor{shadecolor}{RGB}{248,248,248}
\newenvironment{Shaded}{\begin{snugshade}}{\end{snugshade}}
\newcommand{\AlertTok}[1]{\textcolor[rgb]{0.94,0.16,0.16}{#1}}
\newcommand{\AnnotationTok}[1]{\textcolor[rgb]{0.56,0.35,0.01}{\textbf{\textit{#1}}}}
\newcommand{\AttributeTok}[1]{\textcolor[rgb]{0.77,0.63,0.00}{#1}}
\newcommand{\BaseNTok}[1]{\textcolor[rgb]{0.00,0.00,0.81}{#1}}
\newcommand{\BuiltInTok}[1]{#1}
\newcommand{\CharTok}[1]{\textcolor[rgb]{0.31,0.60,0.02}{#1}}
\newcommand{\CommentTok}[1]{\textcolor[rgb]{0.56,0.35,0.01}{\textit{#1}}}
\newcommand{\CommentVarTok}[1]{\textcolor[rgb]{0.56,0.35,0.01}{\textbf{\textit{#1}}}}
\newcommand{\ConstantTok}[1]{\textcolor[rgb]{0.00,0.00,0.00}{#1}}
\newcommand{\ControlFlowTok}[1]{\textcolor[rgb]{0.13,0.29,0.53}{\textbf{#1}}}
\newcommand{\DataTypeTok}[1]{\textcolor[rgb]{0.13,0.29,0.53}{#1}}
\newcommand{\DecValTok}[1]{\textcolor[rgb]{0.00,0.00,0.81}{#1}}
\newcommand{\DocumentationTok}[1]{\textcolor[rgb]{0.56,0.35,0.01}{\textbf{\textit{#1}}}}
\newcommand{\ErrorTok}[1]{\textcolor[rgb]{0.64,0.00,0.00}{\textbf{#1}}}
\newcommand{\ExtensionTok}[1]{#1}
\newcommand{\FloatTok}[1]{\textcolor[rgb]{0.00,0.00,0.81}{#1}}
\newcommand{\FunctionTok}[1]{\textcolor[rgb]{0.00,0.00,0.00}{#1}}
\newcommand{\ImportTok}[1]{#1}
\newcommand{\InformationTok}[1]{\textcolor[rgb]{0.56,0.35,0.01}{\textbf{\textit{#1}}}}
\newcommand{\KeywordTok}[1]{\textcolor[rgb]{0.13,0.29,0.53}{\textbf{#1}}}
\newcommand{\NormalTok}[1]{#1}
\newcommand{\OperatorTok}[1]{\textcolor[rgb]{0.81,0.36,0.00}{\textbf{#1}}}
\newcommand{\OtherTok}[1]{\textcolor[rgb]{0.56,0.35,0.01}{#1}}
\newcommand{\PreprocessorTok}[1]{\textcolor[rgb]{0.56,0.35,0.01}{\textit{#1}}}
\newcommand{\RegionMarkerTok}[1]{#1}
\newcommand{\SpecialCharTok}[1]{\textcolor[rgb]{0.00,0.00,0.00}{#1}}
\newcommand{\SpecialStringTok}[1]{\textcolor[rgb]{0.31,0.60,0.02}{#1}}
\newcommand{\StringTok}[1]{\textcolor[rgb]{0.31,0.60,0.02}{#1}}
\newcommand{\VariableTok}[1]{\textcolor[rgb]{0.00,0.00,0.00}{#1}}
\newcommand{\VerbatimStringTok}[1]{\textcolor[rgb]{0.31,0.60,0.02}{#1}}
\newcommand{\WarningTok}[1]{\textcolor[rgb]{0.56,0.35,0.01}{\textbf{\textit{#1}}}}
\usepackage{graphicx,grffile}
\makeatletter
\def\maxwidth{\ifdim\Gin@nat@width>\linewidth\linewidth\else\Gin@nat@width\fi}
\def\maxheight{\ifdim\Gin@nat@height>\textheight\textheight\else\Gin@nat@height\fi}
\makeatother
% Scale images if necessary, so that they will not overflow the page
% margins by default, and it is still possible to overwrite the defaults
% using explicit options in \includegraphics[width, height, ...]{}
\setkeys{Gin}{width=\maxwidth,height=\maxheight,keepaspectratio}
\IfFileExists{parskip.sty}{%
\usepackage{parskip}
}{% else
\setlength{\parindent}{0pt}
\setlength{\parskip}{6pt plus 2pt minus 1pt}
}
\setlength{\emergencystretch}{3em}  % prevent overfull lines
\providecommand{\tightlist}{%
  \setlength{\itemsep}{0pt}\setlength{\parskip}{0pt}}
\setcounter{secnumdepth}{0}
% Redefines (sub)paragraphs to behave more like sections
\ifx\paragraph\undefined\else
\let\oldparagraph\paragraph
\renewcommand{\paragraph}[1]{\oldparagraph{#1}\mbox{}}
\fi
\ifx\subparagraph\undefined\else
\let\oldsubparagraph\subparagraph
\renewcommand{\subparagraph}[1]{\oldsubparagraph{#1}\mbox{}}
\fi

%%% Use protect on footnotes to avoid problems with footnotes in titles
\let\rmarkdownfootnote\footnote%
\def\footnote{\protect\rmarkdownfootnote}

%%% Change title format to be more compact
\usepackage{titling}

% Create subtitle command for use in maketitle
\providecommand{\subtitle}[1]{
  \posttitle{
    \begin{center}\large#1\end{center}
    }
}

\setlength{\droptitle}{-2em}

  \title{Ensemble Methods for Classification}
    \pretitle{\vspace{\droptitle}\centering\huge}
  \posttitle{\par}
  \subtitle{Majority Vote and Stacking}
  \author{}
    \preauthor{}\postauthor{}
    \date{}
    \predate{}\postdate{}
  
\usepackage{booktabs}
\usepackage{multicol}

\begin{document}
\maketitle

\hypertarget{ensembles-example-ionosphere-radar-data}{%
\subsection{Ensembles Example: Ionosphere radar
data}\label{ensembles-example-ionosphere-radar-data}}

This example is adapted from a discussion at
\url{https://burakhimmetoglu.com/2016/12/01/stacking-models-for-improved-predictions/}

Our data set for today is published and described at
\url{https://archive.ics.uci.edu/ml/datasets/ionosphere}:

\begin{quote}
This radar data was collected by a system in Goose Bay, Labrador. This
system consists of a phased array of 16 high-frequency antennas with a
total transmitted power on the order of 6.4 kilowatts. See the paper for
more details. The targets were free electrons in the ionosphere.
``Good'' radar returns are those showing evidence of some type of
structure in the ionosphere. ``Bad'' returns are those that do not;
their signals pass through the ionosphere.

Received signals were processed using an autocorrelation function whose
arguments are the time of a pulse and the pulse number. There were 17
pulse numbers for the Goose Bay system. Instances in this databse are
described by 2 attributes per pulse number, corresponding to the complex
values returned by the function resulting from the complex
electromagnetic signal.

Attribute Information:

\begin{itemize}
\tightlist
\item
  All 34 are continuous
\item
  The 35th attribute is either ``good'' or ``bad'' according to the
  definition summarized above. This is a binary classification task.
\end{itemize}
\end{quote}

\begin{Shaded}
\begin{Highlighting}[]
\KeywordTok{library}\NormalTok{(readr)}
\KeywordTok{library}\NormalTok{(dplyr)}
\KeywordTok{library}\NormalTok{(ggplot2)}
\KeywordTok{library}\NormalTok{(gridExtra)}
\KeywordTok{library}\NormalTok{(purrr)}
\KeywordTok{library}\NormalTok{(glmnet)}
\KeywordTok{library}\NormalTok{(caret)}

\CommentTok{# read in data}
\NormalTok{ionosphere <-}\StringTok{ }\KeywordTok{read_csv}\NormalTok{(}\StringTok{"http://www.evanlray.com/data/UCIML/ionosphere/ionosphere.data"}\NormalTok{, }\DataTypeTok{col_names =} \OtherTok{FALSE}\NormalTok{)}

\CommentTok{# X2 was all 0's}
\NormalTok{ionosphere <-}\StringTok{ }\NormalTok{ionosphere }\OperatorTok\StringTok{ }\KeywordTok{select}\NormalTok{(}\OperatorTok{-}\NormalTok{X2)}

\CommentTok{# Convert prediction target to a factor}
\NormalTok{ionosphere}\OperatorTok{$}\NormalTok{X35 <-}\StringTok{ }\KeywordTok{factor}\NormalTok{(ionosphere}\OperatorTok{$}\NormalTok{X35)}


\CommentTok{## Initial train/test split ("estimation"/test) and cross-validation folds}
\KeywordTok{set.seed}\NormalTok{(}\DecValTok{63770}\NormalTok{)}
\NormalTok{tt_inds <-}\StringTok{ }\NormalTok{caret}\OperatorTok{::}\KeywordTok{createDataPartition}\NormalTok{(ionosphere}\OperatorTok{$}\NormalTok{X35, }\DataTypeTok{p =} \FloatTok{0.8}\NormalTok{)}
\NormalTok{ionosphere_train <-}\StringTok{ }\NormalTok{ionosphere }\OperatorTok\StringTok{ }\KeywordTok{slice}\NormalTok{(tt_inds[[}\DecValTok{1}\NormalTok{]])}
\NormalTok{ionosphere_test <-}\StringTok{ }\NormalTok{ionosphere }\OperatorTok\StringTok{ }\KeywordTok{slice}\NormalTok{(}\OperatorTok{-}\NormalTok{tt_inds[[}\DecValTok{1}\NormalTok{]])}

\NormalTok{crossval_val_fold_inds <-}\StringTok{ }\NormalTok{caret}\OperatorTok{::}\KeywordTok{createFolds}\NormalTok{(}
  \DataTypeTok{y =}\NormalTok{ ionosphere_train}\OperatorTok{$}\NormalTok{X35, }\CommentTok{# response variable as a vector}
  \DataTypeTok{k =} \DecValTok{10} \CommentTok{# number of folds for cross-validation}
\NormalTok{)}

\NormalTok{get_complementary_inds <-}\StringTok{ }\ControlFlowTok{function}\NormalTok{(x) \{}
  \KeywordTok{return}\NormalTok{(}\KeywordTok{seq_len}\NormalTok{(}\KeywordTok{nrow}\NormalTok{(ionosphere_train))[}\OperatorTok{-}\NormalTok{x])}
\NormalTok{\}}
\NormalTok{crossval_train_fold_inds <-}\StringTok{ }\KeywordTok{map}\NormalTok{(crossval_val_fold_inds, get_complementary_inds)}
\end{Highlighting}
\end{Shaded}

\newpage

\hypertarget{individual-methods}{%
\subsubsection{Individual Methods}\label{individual-methods}}

\hypertarget{logistic-regression}{%
\paragraph{Logistic Regression}\label{logistic-regression}}

\begin{Shaded}
\begin{Highlighting}[]
\NormalTok{logistic_fit <-}\StringTok{ }\KeywordTok{train}\NormalTok{(}
  \DataTypeTok{form =}\NormalTok{ X35 }\OperatorTok{~}\StringTok{ }\NormalTok{.,}
  \DataTypeTok{data =}\NormalTok{ ionosphere_train,}
  \DataTypeTok{family =} \StringTok{"binomial"}\NormalTok{, }\CommentTok{# this is an argument to glm}
  \DataTypeTok{method =} \StringTok{"glm"}\NormalTok{, }\CommentTok{# method for fit}
  \DataTypeTok{trControl =} \KeywordTok{trainControl}\NormalTok{(}\DataTypeTok{method =} \StringTok{"cv"}\NormalTok{, }\CommentTok{# evaluate method performance via cross-validation}
    \DataTypeTok{number =} \DecValTok{10}\NormalTok{, }\CommentTok{# number of folds for cross-validation}
    \DataTypeTok{index =}\NormalTok{ crossval_train_fold_inds, }\CommentTok{# I'm specifying which folds to use, for consistency across methods}
    \DataTypeTok{indexOut =}\NormalTok{ crossval_val_fold_inds, }\CommentTok{# I'm specifying which folds to use, for consistency across methods}
    \DataTypeTok{returnResamp =} \StringTok{"all"}\NormalTok{, }\CommentTok{# return information from cross-validation}
    \DataTypeTok{savePredictions =} \OtherTok{TRUE}\NormalTok{, }\CommentTok{# return validation set predictions from cross-validation}
    \DataTypeTok{classProbs =} \OtherTok{TRUE}\NormalTok{) }\CommentTok{# return validation set predicted class probabilities from cross-validation}
\NormalTok{)}

\NormalTok{logistic_fit}\OperatorTok{$}\NormalTok{results}
\end{Highlighting}
\end{Shaded}

\begin{verbatim}
##   parameter  Accuracy     Kappa AccuracySD   KappaSD
## 1      none 0.8504926 0.6652007 0.07110272 0.1659337
\end{verbatim}

\begin{Shaded}
\begin{Highlighting}[]
\KeywordTok{head}\NormalTok{(logistic_fit}\OperatorTok{$}\NormalTok{pred)}
\end{Highlighting}
\end{Shaded}

\begin{verbatim}
##   pred obs            b            g rowIndex parameter Resample
## 1    g   g 2.052430e-01 7.947570e-01        1      none   Fold01
## 2    g   g 2.576157e-02 9.742384e-01       10      none   Fold01
## 3    g   b 3.994962e-06 9.999960e-01       13      none   Fold01
## 4    g   g 3.042912e-04 9.996957e-01       16      none   Fold01
## 5    g   g 1.427377e-03 9.985726e-01       17      none   Fold01
## 6    b   b 1.000000e+00 1.735932e-11       38      none   Fold01
\end{verbatim}

\begin{Shaded}
\begin{Highlighting}[]
\NormalTok{logistic_fit}\OperatorTok{$}\NormalTok{pred }\OperatorTok
\StringTok{  }\KeywordTok{group_by}\NormalTok{(Resample) }\OperatorTok
\StringTok{  }\KeywordTok{summarize}\NormalTok{(}\DataTypeTok{accuracy =} \KeywordTok{mean}\NormalTok{(pred }\OperatorTok{==}\StringTok{ }\NormalTok{obs)) }\OperatorTok
\StringTok{  }\KeywordTok{ungroup}\NormalTok{() }\OperatorTok
\StringTok{  }\KeywordTok{summarize}\NormalTok{(}\DataTypeTok{accuracy =} \KeywordTok{mean}\NormalTok{(accuracy))}
\end{Highlighting}
\end{Shaded}

\begin{verbatim}
## # A tibble: 1 x 1
##   accuracy
##      <dbl>
## 1    0.850
\end{verbatim}

\newpage

\hypertarget{knn}{%
\paragraph{KNN}\label{knn}}

\begin{Shaded}
\begin{Highlighting}[]
\NormalTok{knn_fit <-}\StringTok{ }\KeywordTok{train}\NormalTok{(}
  \DataTypeTok{form =}\NormalTok{ X35 }\OperatorTok{~}\StringTok{ }\NormalTok{.,}
  \DataTypeTok{data =}\NormalTok{ ionosphere_train,}
  \DataTypeTok{method =} \StringTok{"knn"}\NormalTok{,}
  \DataTypeTok{preProcess =} \StringTok{"scale"}\NormalTok{,}
  \DataTypeTok{trControl =} \KeywordTok{trainControl}\NormalTok{(}\DataTypeTok{method =} \StringTok{"cv"}\NormalTok{,}
    \DataTypeTok{number =} \DecValTok{10}\NormalTok{,}
    \DataTypeTok{index =}\NormalTok{ crossval_train_fold_inds, }\CommentTok{# I'm specifying which folds to use, for consistency across methods}
    \DataTypeTok{indexOut =}\NormalTok{ crossval_val_fold_inds, }\CommentTok{# I'm specifying which folds to use, for consistency across methods}
    \DataTypeTok{returnResamp =} \StringTok{"all"}\NormalTok{,}
    \DataTypeTok{savePredictions =} \OtherTok{TRUE}\NormalTok{,}
    \DataTypeTok{classProbs =} \OtherTok{TRUE}\NormalTok{),}
  \DataTypeTok{tuneGrid =} \KeywordTok{data.frame}\NormalTok{(}\DataTypeTok{k =} \DecValTok{1}\OperatorTok{:}\DecValTok{20}\NormalTok{)}
\NormalTok{)}

\NormalTok{knn_fit}\OperatorTok{$}\NormalTok{results}
\end{Highlighting}
\end{Shaded}

\begin{verbatim}
##     k  Accuracy     Kappa AccuracySD   KappaSD
## 1   1 0.8540640 0.6570214 0.08983538 0.2252713
## 2   2 0.8362069 0.6136325 0.07200427 0.1706708
## 3   3 0.8147783 0.5538864 0.09519951 0.2273648
## 4   4 0.8041872 0.5226215 0.09867257 0.2402436
## 5   5 0.8219212 0.5680881 0.09692917 0.2403482
## 6   6 0.8254926 0.5792690 0.08680310 0.2128852
## 7   7 0.8040640 0.5219281 0.06620901 0.1715432
## 8   8 0.8076355 0.5324729 0.06163325 0.1578542
## 9   9 0.8041872 0.5237628 0.05671862 0.1452014
## 10 10 0.8075123 0.5300637 0.06400532 0.1680257
## 11 11 0.7932266 0.4880301 0.05349060 0.1460922
## 12 12 0.7896552 0.4784264 0.04981489 0.1364785
## 13 13 0.8003695 0.5071866 0.05702811 0.1567114
## 14 14 0.8003695 0.5082105 0.05182000 0.1414887
## 15 15 0.7896552 0.4785748 0.06466996 0.1707640
## 16 16 0.7754926 0.4389018 0.05922736 0.1593515
## 17 17 0.7754926 0.4383273 0.06157375 0.1677752
## 18 18 0.7684729 0.4133543 0.07438714 0.2072648
## 19 19 0.7647783 0.4022990 0.07254520 0.2019538
## 20 20 0.7472906 0.3506893 0.04908562 0.1432625
\end{verbatim}

\newpage

\hypertarget{trees}{%
\paragraph{Trees}\label{trees}}

\begin{Shaded}
\begin{Highlighting}[]
\NormalTok{rpart_fit <-}\StringTok{ }\KeywordTok{train}\NormalTok{(}
  \DataTypeTok{form =}\NormalTok{ X35 }\OperatorTok{~}\StringTok{ }\NormalTok{.,}
  \DataTypeTok{data =}\NormalTok{ ionosphere_train,}
  \DataTypeTok{method =} \StringTok{"rpart"}\NormalTok{,}
  \DataTypeTok{trControl =} \KeywordTok{trainControl}\NormalTok{(}\DataTypeTok{method =} \StringTok{"cv"}\NormalTok{,}
    \DataTypeTok{number =} \DecValTok{10}\NormalTok{,}
    \DataTypeTok{index =}\NormalTok{ crossval_train_fold_inds, }\CommentTok{# I'm specifying which folds to use, for consistency across methods}
    \DataTypeTok{indexOut =}\NormalTok{ crossval_val_fold_inds, }\CommentTok{# I'm specifying which folds to use, for consistency across methods}
    \DataTypeTok{returnResamp =} \StringTok{"all"}\NormalTok{,}
    \DataTypeTok{savePredictions =} \OtherTok{TRUE}\NormalTok{,}
    \DataTypeTok{classProbs =} \OtherTok{TRUE}\NormalTok{),}
  \DataTypeTok{tuneGrid =} \KeywordTok{data.frame}\NormalTok{(}\DataTypeTok{cp =} \KeywordTok{seq}\NormalTok{(}\DataTypeTok{from =} \DecValTok{0}\NormalTok{, }\DataTypeTok{to =} \DecValTok{1}\NormalTok{, }\DataTypeTok{length =} \DecValTok{20}\NormalTok{))}
\NormalTok{)}

\NormalTok{rpart_fit}\OperatorTok{$}\NormalTok{results}
\end{Highlighting}
\end{Shaded}

\begin{verbatim}
##            cp  Accuracy      Kappa  AccuracySD    KappaSD
## 1  0.00000000 0.8543103 0.67345204 0.058505963 0.13243804
## 2  0.05263158 0.8828818 0.74337870 0.057119643 0.12038235
## 3  0.10526316 0.8828818 0.74337870 0.057119643 0.12038235
## 4  0.15789474 0.8828818 0.74337870 0.057119643 0.12038235
## 5  0.21052632 0.8364532 0.61862568 0.069272228 0.19298353
## 6  0.26315789 0.7795567 0.46022616 0.045605318 0.11811999
## 7  0.31578947 0.7795567 0.46022616 0.045605318 0.11811999
## 8  0.36842105 0.7795567 0.46022616 0.045605318 0.11811999
## 9  0.42105263 0.7795567 0.46022616 0.045605318 0.11811999
## 10 0.47368421 0.7795567 0.46022616 0.045605318 0.11811999
## 11 0.52631579 0.6477833 0.02432432 0.024382992 0.07692027
## 12 0.57894737 0.6406404 0.00000000 0.007009975 0.00000000
## 13 0.63157895 0.6406404 0.00000000 0.007009975 0.00000000
## 14 0.68421053 0.6406404 0.00000000 0.007009975 0.00000000
## 15 0.73684211 0.6406404 0.00000000 0.007009975 0.00000000
## 16 0.78947368 0.6406404 0.00000000 0.007009975 0.00000000
## 17 0.84210526 0.6406404 0.00000000 0.007009975 0.00000000
## 18 0.89473684 0.6406404 0.00000000 0.007009975 0.00000000
## 19 0.94736842 0.6406404 0.00000000 0.007009975 0.00000000
## 20 1.00000000 0.6406404 0.00000000 0.007009975 0.00000000
\end{verbatim}

\hypertarget{test-set-predictions-from-each-of-the-3-methods-above}{%
\paragraph{Test set predictions from each of the 3 methods
above:}\label{test-set-predictions-from-each-of-the-3-methods-above}}

\begin{Shaded}
\begin{Highlighting}[]
\NormalTok{logistic_preds <-}\StringTok{ }\KeywordTok{predict}\NormalTok{(logistic_fit, }\DataTypeTok{newdata =}\NormalTok{ ionosphere_test)}
\KeywordTok{mean}\NormalTok{(ionosphere_test}\OperatorTok{$}\NormalTok{X35 }\OperatorTok{!=}\StringTok{ }\NormalTok{logistic_preds)}
\end{Highlighting}
\end{Shaded}

\begin{verbatim}
## [1] 0.07142857
\end{verbatim}

\begin{Shaded}
\begin{Highlighting}[]
\NormalTok{knn_preds <-}\StringTok{ }\KeywordTok{predict}\NormalTok{(knn_fit, }\DataTypeTok{newdata =}\NormalTok{ ionosphere_test)}
\KeywordTok{mean}\NormalTok{(ionosphere_test}\OperatorTok{$}\NormalTok{X35 }\OperatorTok{!=}\StringTok{ }\NormalTok{knn_preds)}
\end{Highlighting}
\end{Shaded}

\begin{verbatim}
## [1] 0.08571429
\end{verbatim}

\begin{Shaded}
\begin{Highlighting}[]
\NormalTok{rpart_preds <-}\StringTok{ }\KeywordTok{predict}\NormalTok{(rpart_fit, }\DataTypeTok{newdata =}\NormalTok{ ionosphere_test)}
\KeywordTok{mean}\NormalTok{(ionosphere_test}\OperatorTok{$}\NormalTok{X35 }\OperatorTok{!=}\StringTok{ }\NormalTok{rpart_preds)}
\end{Highlighting}
\end{Shaded}

\begin{verbatim}
## [1] 0.05714286
\end{verbatim}

\newpage

\hypertarget{ensemble-methods}{%
\subsubsection{Ensemble Methods}\label{ensemble-methods}}

\hypertarget{majority-vote}{%
\paragraph{Majority Vote}\label{majority-vote}}

\begin{Shaded}
\begin{Highlighting}[]
\NormalTok{majority_vote_preds <-}\StringTok{ }\KeywordTok{ifelse}\NormalTok{(}
\NormalTok{  (logistic_preds }\OperatorTok{==}\StringTok{ "g"}\NormalTok{) }\OperatorTok{+}\StringTok{ }\NormalTok{(knn_preds }\OperatorTok{==}\StringTok{ "g"}\NormalTok{) }\OperatorTok{+}\StringTok{ }\NormalTok{(rpart_preds }\OperatorTok{==}\StringTok{ "g"}\NormalTok{) }\OperatorTok{>=}\StringTok{ }\DecValTok{2}\NormalTok{,}
  \StringTok{"g"}\NormalTok{,}
  \StringTok{"b"}
\NormalTok{)}

\KeywordTok{mean}\NormalTok{(ionosphere_test}\OperatorTok{$}\NormalTok{X35 }\OperatorTok{!=}\StringTok{ }\NormalTok{majority_vote_preds)}
\end{Highlighting}
\end{Shaded}

\begin{verbatim}
## [1] 0.02857143
\end{verbatim}

\hypertarget{mean-of-class-probabilities-from-individual-methods}{%
\paragraph{Mean of Class Probabilities from Individual
Methods}\label{mean-of-class-probabilities-from-individual-methods}}

\begin{Shaded}
\begin{Highlighting}[]
\NormalTok{logistic_class_probs <-}\StringTok{ }\KeywordTok{predict}\NormalTok{(logistic_fit, }\DataTypeTok{newdata =}\NormalTok{ ionosphere_test, }\DataTypeTok{type =} \StringTok{"prob"}\NormalTok{)}
\NormalTok{logistic_prob_g <-}\StringTok{ }\NormalTok{logistic_class_probs[, }\DecValTok{2}\NormalTok{]}

\NormalTok{knn_class_probs <-}\StringTok{ }\KeywordTok{predict}\NormalTok{(knn_fit, }\DataTypeTok{newdata =}\NormalTok{ ionosphere_test, }\DataTypeTok{type =} \StringTok{"prob"}\NormalTok{)}
\NormalTok{knn_prob_g <-}\StringTok{ }\NormalTok{knn_class_probs[, }\DecValTok{2}\NormalTok{]}

\NormalTok{rpart_class_probs <-}\StringTok{ }\KeywordTok{predict}\NormalTok{(rpart_fit, }\DataTypeTok{newdata =}\NormalTok{ ionosphere_test, }\DataTypeTok{type =} \StringTok{"prob"}\NormalTok{)}
\NormalTok{rpart_prob_g <-}\StringTok{ }\NormalTok{rpart_class_probs[, }\DecValTok{2}\NormalTok{]}

\NormalTok{mean_prob_g <-}\StringTok{ }\NormalTok{(logistic_prob_g }\OperatorTok{+}\StringTok{ }\NormalTok{knn_prob_g }\OperatorTok{+}\StringTok{ }\NormalTok{rpart_prob_g) }\OperatorTok{/}\StringTok{ }\DecValTok{3}
\NormalTok{mean_prob_preds <-}\StringTok{ }\KeywordTok{ifelse}\NormalTok{(}
\NormalTok{  mean_prob_g }\OperatorTok{>=}\StringTok{ }\FloatTok{0.5}\NormalTok{,}
  \StringTok{"g"}\NormalTok{,}
  \StringTok{"b"}
\NormalTok{)}
\KeywordTok{mean}\NormalTok{(ionosphere_test}\OperatorTok{$}\NormalTok{X35 }\OperatorTok{!=}\StringTok{ }\NormalTok{mean_prob_preds)}
\end{Highlighting}
\end{Shaded}

\begin{verbatim}
## [1] 0.04285714
\end{verbatim}

\newpage

\hypertarget{stacking-fit-a-model-to-combine-predicted-class-membership-probabilities}{%
\paragraph{Stacking: Fit a model to combine predicted class membership
probabilities}\label{stacking-fit-a-model-to-combine-predicted-class-membership-probabilities}}

\textbf{Process:}

Estimation:

\begin{enumerate}
\def\labelenumi{\arabic{enumi}.}
\tightlist
\item
  Get cross-validated predictions for each ``stage 1'' or ``component''
  model (this was done above)
\item
  Create a new data set where the explanatory variables are the
  cross-validated predictions from the component models
\item
  Fit a ``stage 2'' model to predict the response based on the component
  model predictions
\end{enumerate}

Prediction for test set:

\begin{enumerate}
\def\labelenumi{\arabic{enumi}.}
\setcounter{enumi}{3}
\tightlist
\item
  Re-fit each component model to the full training data set, make
  predictions for the test set from each component model (this was done
  above)
\item
  Create a new data set where the explanatory variables are the test set
  predictions from the component models
\item
  Predict using the stage 2 model fit from step 3 and the data frame
  created in step 5.
\end{enumerate}

\begin{Shaded}
\begin{Highlighting}[]
\CommentTok{# Step 2: data set with component model predictions as explanatory variables}
\NormalTok{ionosphere_train <-}\StringTok{ }\NormalTok{ionosphere_train }\OperatorTok
\StringTok{  }\KeywordTok{mutate}\NormalTok{(}
    \DataTypeTok{logistic_prob_g =}\NormalTok{ logistic_fit}\OperatorTok{$}\NormalTok{pred }\OperatorTok
\StringTok{      }\KeywordTok{arrange}\NormalTok{(rowIndex) }\OperatorTok
\StringTok{      }\KeywordTok{pull}\NormalTok{(g),}
    \DataTypeTok{knn_prob_g =}\NormalTok{ knn_fit}\OperatorTok{$}\NormalTok{pred }\OperatorTok
\StringTok{      }\KeywordTok{filter}\NormalTok{(k }\OperatorTok{==}\StringTok{ }\DecValTok{1}\NormalTok{) }\OperatorTok
\StringTok{      }\KeywordTok{arrange}\NormalTok{(rowIndex) }\OperatorTok
\StringTok{      }\KeywordTok{pull}\NormalTok{(g),}
    \DataTypeTok{rpart_prob_g =}\NormalTok{ rpart_fit}\OperatorTok{$}\NormalTok{pred }\OperatorTok
\StringTok{      }\KeywordTok{filter}\NormalTok{(cp }\OperatorTok{==}\StringTok{ }\NormalTok{rpart_fit}\OperatorTok{$}\NormalTok{bestTune}\OperatorTok{$}\NormalTok{cp) }\OperatorTok
\StringTok{      }\KeywordTok{arrange}\NormalTok{(rowIndex) }\OperatorTok
\StringTok{      }\KeywordTok{pull}\NormalTok{(g)}
\NormalTok{  )}

\CommentTok{# Step 3: fit model using component model predictions as explanatory variables}
\NormalTok{stacking_logistic_fit <-}\StringTok{ }\KeywordTok{train}\NormalTok{(}
  \DataTypeTok{form =}\NormalTok{ X35 }\OperatorTok{~}\StringTok{ }\NormalTok{logistic_prob_g }\OperatorTok{+}\StringTok{ }\NormalTok{knn_prob_g }\OperatorTok{+}\StringTok{ }\NormalTok{rpart_prob_g,}
  \DataTypeTok{data =}\NormalTok{ ionosphere_train,}
  \DataTypeTok{family =} \StringTok{"binomial"}\NormalTok{,}
  \DataTypeTok{method =} \StringTok{"glm"}
\NormalTok{)}

\CommentTok{# Step 5: Assemble data frame of test set predictions from each component model}
\NormalTok{stacking_test_x <-}\StringTok{ }\KeywordTok{data.frame}\NormalTok{(}
  \DataTypeTok{logistic_prob_g =}\NormalTok{ logistic_prob_g,}
  \DataTypeTok{knn_prob_g =}\NormalTok{ knn_prob_g,}
  \DataTypeTok{rpart_prob_g =}\NormalTok{ rpart_prob_g}
\NormalTok{)}

\CommentTok{# Step 6: Stacked model predictions}
\NormalTok{stacking_preds <-}\StringTok{ }\KeywordTok{predict}\NormalTok{(stacking_logistic_fit, stacking_test_x)}

\CommentTok{# Calculate error rate}
\KeywordTok{mean}\NormalTok{(ionosphere_test}\OperatorTok{$}\NormalTok{X35 }\OperatorTok{!=}\StringTok{ }\NormalTok{stacking_preds)}
\end{Highlighting}
\end{Shaded}

\begin{verbatim}
## [1] 0.01428571
\end{verbatim}

\newpage

\hypertarget{stacking-via-knn}{%
\paragraph{Stacking via KNN}\label{stacking-via-knn}}

\begin{itemize}
\tightlist
\item
  We could also use other methods for the second stage model.
\end{itemize}

\begin{Shaded}
\begin{Highlighting}[]
\NormalTok{ionosphere_train <-}\StringTok{ }\NormalTok{ionosphere_train }\OperatorTok
\StringTok{  }\KeywordTok{mutate}\NormalTok{(}
    \DataTypeTok{logistic_prob_g =}\NormalTok{ logistic_fit}\OperatorTok{$}\NormalTok{pred }\OperatorTok
\StringTok{      }\KeywordTok{arrange}\NormalTok{(rowIndex) }\OperatorTok
\StringTok{      }\KeywordTok{pull}\NormalTok{(g),}
    \DataTypeTok{knn_prob_g =}\NormalTok{ knn_fit}\OperatorTok{$}\NormalTok{pred }\OperatorTok
\StringTok{      }\KeywordTok{filter}\NormalTok{(k }\OperatorTok{==}\StringTok{ }\DecValTok{1}\NormalTok{) }\OperatorTok
\StringTok{      }\KeywordTok{arrange}\NormalTok{(rowIndex) }\OperatorTok
\StringTok{      }\KeywordTok{pull}\NormalTok{(g),}
    \DataTypeTok{rpart_prob_g =}\NormalTok{ rpart_fit}\OperatorTok{$}\NormalTok{pred }\OperatorTok
\StringTok{      }\KeywordTok{filter}\NormalTok{(cp }\OperatorTok{==}\StringTok{ }\NormalTok{rpart_fit}\OperatorTok{$}\NormalTok{bestTune}\OperatorTok{$}\NormalTok{cp) }\OperatorTok
\StringTok{      }\KeywordTok{arrange}\NormalTok{(rowIndex) }\OperatorTok
\StringTok{      }\KeywordTok{pull}\NormalTok{(g)}
\NormalTok{  )}

\NormalTok{stacking_knn_fit <-}\StringTok{ }\KeywordTok{train}\NormalTok{(}
  \DataTypeTok{form =}\NormalTok{ X35 }\OperatorTok{~}\StringTok{ }\NormalTok{logistic_prob_g }\OperatorTok{+}\StringTok{ }\NormalTok{knn_prob_g }\OperatorTok{+}\StringTok{ }\NormalTok{rpart_prob_g,}
  \DataTypeTok{data =}\NormalTok{ ionosphere_train,}
  \DataTypeTok{method =} \StringTok{"knn"}
\NormalTok{)}

\CommentTok{# Assemble data frame of test set predictions from each}
\CommentTok{# component model (these were obtained in the previous part)}
\NormalTok{stacking_test_x <-}\StringTok{ }\KeywordTok{data.frame}\NormalTok{(}
  \DataTypeTok{logistic_prob_g =}\NormalTok{ logistic_prob_g,}
  \DataTypeTok{knn_prob_g =}\NormalTok{ knn_prob_g,}
  \DataTypeTok{rpart_prob_g =}\NormalTok{ rpart_prob_g}
\NormalTok{)}

\CommentTok{# Stacked model predictions}
\NormalTok{stacking_preds <-}\StringTok{ }\KeywordTok{predict}\NormalTok{(stacking_knn_fit, stacking_test_x)}
\KeywordTok{mean}\NormalTok{(ionosphere_test}\OperatorTok{$}\NormalTok{X35 }\OperatorTok{!=}\StringTok{ }\NormalTok{stacking_preds)}
\end{Highlighting}
\end{Shaded}

\begin{verbatim}
## [1] 0.01428571
\end{verbatim}

\hypertarget{notes-about-relative-performance-of-these-methods}{%
\subsubsection{Notes about relative performance of these
methods}\label{notes-about-relative-performance-of-these-methods}}

\begin{itemize}
\tightlist
\item
  The results with the seed I have set make the stacking approaches look
  amazing - but in different runs I did as I was developing this,
  relative performance of the ensemble approaches here varied.
\item
  In general, stacking is the best of these ensemble approaches in terms
  of expected value of performance, if the methods' performance is not
  all equal and we have enough training set data.
\item
  Note that in this example we had only 3 stage 1/component models. In
  general, ensembles are most useful when we have:

  \begin{itemize}
  \tightlist
  \item
    A large number of models that are
    ``diverse''/uncorrelated/predictions are close to independent
  \item
    Enough training data available to reliably tell which component
    models are best and how to combine their predictions effectively.
  \end{itemize}
\item
  Formally, the stacking procedure I did here is not quite right:

  \begin{itemize}
  \tightlist
  \item
    You shouldn't use the same cross-validation results both to select
    tuning parameters like K for KNN and cp for classification trees,
    AND as inputs to stacking.
  \item
    Doing this means models that overfit validation data will get too
    much weight
  \item
    In practice if you're selecting one tuning parameter this shouldn't
    matter too much.
  \end{itemize}
\end{itemize}


\end{document}
